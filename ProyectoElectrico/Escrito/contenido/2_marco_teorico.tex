% ----------------------
  \chapter{Marco Teorico} 
% ----------------------
\label{C:antecedentes}

\section{Introducción}
En el ARCOS-Lab se lleva a cabo la construcción de un robot humanoide de servicio a las personas. 
Este debe ser capas de hacer tareas complejas 
alrededor de seres humanos, sin hacerle daño a estos. \\
Para esto el robot deberá poder variar su rigidez dependiendo su entorno. 
Esto es lo que lo diferencia en un robot industrial el cual no variara su 
objetivo sin importar los obstaculos que tenga en el medio. 
Es decir si un humano se mete en el camino de un robot industrial este continuará su 
trayecto sin detenerse, provocando traumas fisicos en el humano. 
En cambio, el robot humanoide del ARCOS-Lab tendrá la capacidad de sensar cuando un 
obstaculo esta en el medio y variar su rigidez para no provocarle un daño a este.\\
El robot humanoide en este momento cuenta con una base omnidireccional, 2 brazos robóticos KUKA LWR 4+, una mano robotica multisensorial DLR HIT II y 
una cabeza robotica creada en este laboratorio. 
Lo que le haria falta seria el torso y hombros roboticos para poder juntar todos los componentes de este.\\
Para lograr un control suave con el ambiente es necesario implementarle un control por impedancia el cual podra sensar fuerzas externas y modificar su 
rigidez dependiendo de estas. 
Para esto se explicará que es el control por impedancia y como implementarlo.

\section{Robots Humanoides}
Los robots humanoides son aquellos que semejan al ser humano. 
Generalmente estos  poseen una cabeza, 2 brazos, 2 manos, un torso y 2 piernas. 
Aunque algunas formas de robots humanoides pueden modelar sólo una parte del cuerpo. 
Por ejemplo el robot humanoide del ARCOS-Lab posee una base 
omnidireccional en vez de 2 piernas bipedas. \\
En un futuro estos robots seran capaces de hacer tareas de alto riesgo para no poner al ser humano en peligro. 
Por ejemplo ayudar a los bomberos en incendios o trabajo en zonas radioactivas. 
Al mismo tiempo tendran que resolver tareas cotidianas para la ayuda de personas con discapacidades, 
adultos mayores y personas que necesiten de asistencia. Para esto el robot deberá ser capaz de hacer cocinar, limpiar e interactuar con el humano. 
Para estas acciones se requiere de una gran precision, ya que para cocinar un desayuno basico es necesario poder manipular un huevo sin romperlo por 
accidente.
Como se puede observar en la tesis de Israel Chaves  \tesis existen en la actualidad varios robots humanoides, como el TUM, 


\section{Torso}


\section{Control por impedancia}
Cada vez que se le hace una fuerza a un objeto este se deformara microscopicamente. Puede ser hasta con solo tocar una barra metalica esta se doblará
aunque no sea detectable a la vista humana. Para detectar estas deformaciones se utilizan galgas extensiometricas, las cuales tienen una resistencia 
que varia según cuanto se estiren o encojan. Estas son muy sensibles a cambios de longuitud, por lo que con pequeñas deformaciones en el material se
podrán detectar. Para lograr una mejor detección y poderlo sensar es necesario hacer un puente Wheatstone el cual pone 3 resistencias con el mismo valor
más la galga con resistencia variable y se pasa por un par de amplificadores operacionales para que no hayan fugas de corriente y se pueda medir
con mayor exactitud la diferencia de resistencia. Como se puede observar en figura \figura este despues se pasa por otro amplificador operacional 
para potenciar la tensión y tener mejores mediciones. Este circuito se utiliza con un integrado llamado \circuito que este recibe la diferencia de tension
por el puente Wheatstone y lo amplifica para la medicion deseada.


\section{Convenciones básicas de formato en español}


% Revisar qué hay que dejar de aquí:

\section{Más explicaciones aquí}

En el segundo capítulo del informe, debe resumirse el estudio realizado sobre \emph{estado de la técnica}, en la temática relacionada con el proyecto.  Este se puede denominar ``Antecedentes'', ``Marco de referencia'', ``Base teórica'', o ``Marco teórico''.

Por tratarse de una presentación con base en la recopilación, el análisis y la síntesis de trabajos de otros autores, la referencia adecuada a los mismos, es indispensable.  Toda copia (\emph{¡plagio!}), es inaceptable.

Para indicar las fuentes bibliográficas puede utilizarse el comando del paquete \texttt{natbib} \texttt{\textbackslash cite\{bibtexkey\}} para utilizar el formato ``Autor (año)'' o el comando \texttt{\textbackslash citep\{bibtexkey\}} para mostrarla en el formato ``(Autor, año)''.

%en el siguiente párrafo se supone que se han empleado los comandos cite o citep
Por ejemplo se obtiene: ``Según Wang (1996) la frecuencia ...'' (si se cita con \texttt{cite}), o ``... para este diseño se han utilizado modelos determinísticos (Smith, 2005) y estocásticos (Bell et. al, 2010).''  (si se citan con \texttt{citep}).  

El contenido del capítulo debe ser relevante para el proyecto y no ``material de relleno'', o incluido con el único propósito de ``engordar'' el informe.

El estado de la técnica establece el \emph{punto de partida} del estudio realizado y posiblemente también, la \emph{base de comparación} para las pruebas realizadas.

Este capítulo muestra la capacidad de análisis y síntesis del estudiante.
 
%- declaración de una sección ---------------------------------------
\section{Ecuaciones}
Las ecuaciones estarán centradas y numeradas en forma secuencial por capítulo, al margen derecho.  La referencia a ellas se hará utilizando su número.

¡Texto de ejemplo! - ``El modelo utilizado para representar al proceso, es de primer orden más tiempo muerto, dado por la función de transferencia

\begin{equation}  %inclusión de ecuaciones
	P(s) = \frac{K \me^{-Ls}}{Ts+1}, \label{ec:01}
\end{equation}

\noindent donde $K$ es la ganancia, $T$ la ...''  
%si el texto después de la ecuación no inicia un nuevo párrafo y se ha insertado una linea en blanco depues de esta, es necesario poner \noindent para que el texto siguiente no tenga sangría (formato predeterminado).

Las ecuaciones forman parte del texto, por lo que deben terminarse con el signo de puntuación requerido, una coma o un punto.

Para referirse a ellas se hace uso de la etiqueta (\texttt{label}) asignada a la ecuación usando \texttt{\textbackslash eqref\{etiqueta\}} que mostrará su número.  Por ejemplo ``El modelo \eqref{ec:01} es el más utilizado para ...''

%en el siguiente párrafo se supone que se ha utilizado \eqref{etiqueta} para referencias las ecuacioenes
El texto debe mostrar ``... sustituyendo (2.4) y (2.5) en (2.2), se obtiene ...'' y no ``... sustituyendo la ecuación (2.4) y la ecuación (2.5) en la ecuación (2.2), se obtiene...''  

¡Ejemplos de ecuaciones!

Usando \texttt{equation}:
\begin{equation}
	\tau \frac{\md T_{tc}(t)}{\md t} + T_{tc}(t) = T_{gas}(t).
\end{equation}

Ecuaciones alineadas utilizando \texttt{align}:

\begin{align}
	L_1 \frac{\md i_{L_1} (t)}{\md t} &= v(t) - R_1 i_{L_1}(t) - v_c(t), \\
	C \frac{\md v_c (t)}{\md t} &= i_L(t)- \frac{1}{R_2} v_c(t).
\end{align}

\section{Figuras y cuadros}
Las figuras y los cuadros son \emph{elementos flotantes}. Aunque se le puede ``sugerir'' a \LaTeX~ donde ubicarlos, es conveniente dejarlos ``flotar''.

\subsection{Figuras}
Las referencias a las figuras debe hacerse utilizando el número asignado a ellas.  Para esto se le asigna una etiqueta (con \texttt{label}) y luego se utiliza esta para hacer la referencia (con \texttt{ref}).  Usar en el texto el término ``figura'' y no Fig.'' o ``fig.''.

La leyenda (con \texttt{caption}) de la figura, irá en la parte inferior de la misma.  Como en forma predeterminada en la clase \texttt{eieproyecto} las figuras están centradas, no es necesario usar \texttt{centering} para hacerlo.

Por ejemplo ``Considérese el diagrama de bloques mostrado en la figura en donde el proceso controlado está dado por ...''.

No utilizar ``... en la siguiente figura ...'', emplear siempre el número correspondiente para referirse a ellas.

Cuando las figuras son muy pequeñas, se puede colocar la leyenda al lado de la misma, con el ambiente \texttt{SCfigure} del paquete \texttt{sidecap}.  Un ejemplo de esto se muestra en la figura.

Cuando un gráfico muestre varias curvas, estas deben poderse distinguir, no solamente en la pantalla de la computadora, usando diferentes colores, si no también en una impresión en blanco y negro, utilizando lineas de trazos diferentes, como se muestra en la figura.

\LaTeX~ nunca coloca las figuras y los cuadros en una página anterior a la en que son incluidas.  Los elementos flotantes los coloca en la página donde se hace referencia a ellos, o en una de las siguientes.

Además, en el texto debe hacerse referencia a todas las figuras y cuadros incluidos en el informe.  Si alguno de ellos no se menciona en el texto, es que no se requiere para entender el desarrollo presentado y por lo tanto es innecesario y se podría omitir sin que se afecte el informe.

\subsection{Cuadros}
Los cuadros son el otro elemento flotante utilizado en los informes y también es conveniente dejar que \LaTeX~ los coloque en donde considere que es más adecuado.

Los cuadros no llevarán ninguna línea divisoria vertical, solo horizontales. Una en la parte superior (\texttt{toprule}), una bajo la línea de cabecera (\texttt{midrule}) y una en la parte inferior (\texttt{bottomrule}).  Normalmente basta con estas tres líneas, pero si fuera necesaria alguna otra para una división horizontal, esta debe ser del tipo \texttt{midrule}.

Se recomienda revisar los comandos para la construcción de cuadros, incluidos en el manual de la clase \texttt{memoir} \cite{memoir2011}, o en la del paquete \texttt{booktabs} \cite{fear2005}.

La leyenda (\texttt{caption}) del cuadro se mostrará en la parte superior.  Para poder referirse al cuadro (con \texttt{ref}), se le asigna una etiqueta (con \texttt{label}).

En forma predefinida, los cuadros se mostrarán centrados horizontalmente, por lo que no es necesario hacer esa indicación. 

El cuadro \ref{tab:01} es un ejemplo de un cuadro de datos simple.

%inclusión de un cuadro con datos
\begin{table}
\caption{Parámetros de los modelos.} \label{tab:01o}
		\begin{tabular}{@{}*{4}{c}@{}}
    \toprule
    $K_p$ & $T_1$ & $T_2$ & $L$ \\
    \midrule
     1,01 & 1,50 & 0,75 & 0,12 \\
		 1,15 & 2,37 & 0,15 & 0,28 \\
		 2,25 & 5,89 & 2,15 & 1,60 \\
    \bottomrule
    \end{tabular}
\end{table}

Si la primera columna corresponde a leyendas o parámetros que identifican los datos de la línea, esta debe estar justificada a la izquierda, como se muestra en el cuadro \ref{tab:AH}, que ha sido tomada de \cite{astromhagglund2006}.

\begin{table}
\caption{Parámetros de los controladores ...} \label{tab:AH}
\begin{center}
    \begin{tabular}{@{}l*{7}{c}@{}}
    \toprule
    Controller & $K$ & $K_i$ & $K_d$ & $\beta$ & $T_i$ & $T_d$ & IAE \\
    \midrule
    PD &  1,333 & 0 & 1,333 & 1 & 0 &1 & $\infty$ \\
		PI & 0,433 & 0,192 & 0 & 0,14 & 2,25 & 0 & 6,20 \\
		PID MIGO & 1,305 & 0,758 & 1,705 & 0 & 1,72 & 1,31 & 2,25 \\
		PID $T_i=4 \ T_d$ & 1,132 & 0,356 & 0,900 & 0,9 & 3,18 & 0,80 & 2,51 \\
    \bottomrule
    \end{tabular}
\end{center}
\end{table}

Se puede especificar una cabecera para más de una columna y utilizar lineas horizontales que abarquen solo unas pocas columnas, como se muestra en el cuadro \ref{tab:muestra}.

\begin{table}
\caption{Ejemplo de otro cuadro.} \label{tab:muestra}
	\begin{tabular}{@{}l*{4}{c}@{}}
	\toprule
	& \multicolumn{2}{c}{Prueba 1} & \multicolumn{2}{c}{Prueba 2} \\
	\cmidrule(l{2pt}r{2pt}){2-3}\cmidrule(l{2pt}r{2pt}){4-5} 
	& $\Delta E=5$ V & $\Delta E = -5$ V & $\Delta E = 10$ V & $\Delta E = -10$ V \\
	\midrule
	Ganancia             &  1,06 & 0,98 & 1,12 & 0,97 \\
	Tiempo subida, s  &  5,67 & 5,89 & 6,02 & 5,74 \\
	Sobrepaso máx, \%        &  2,67 & 3,25 & 2,91 & 1,56 \\
	Error, \% &  0,25 & 0,56 & 0,97 & 0,18 \\
	\bottomrule
	\end{tabular}
\end{table}

\newpage
Cuando los cuadros son pequeños (abarcan menos de la mitad del ancho del texto), se puede colocar la leyenda a la par del cuadro, utilizando el ambiente \texttt{SCtable} del paquete \texttt{sidecap}, tal como se muestra en el cuadro \ref{tab:01}.  Compare este, con el cuadro \ref{tab:01o}.

\begin{SCtable}
\caption[Parámetros de los modelos]{Parámetros de los modelos, obtenidos a partir de las tres curvas de reacción.} \label{tab:01}
    \begin{tabular}{@{}*{4}{c}@{}}
    \toprule
    $K_p$ & $T_1$ & $T_2$ & $L$ \\
    \midrule
     1,01 & 1,50 & 0,75 & 0,12 \\
		 1,15 & 2,37 & 0,15 & 0,28 \\
		 2,25 & 5,89 & 2,15 & 1,60 \\
    \bottomrule
    \end{tabular}
\end{SCtable}