% ----------------------
  \chapter{Introducción}
% ----------------------
\label{C:introduccion}

%--------------------------------------------------------------------
\section{Introducción del informe}

Los robots de asistencia son creados para ayudar al humano en diferentes tipos de tareas. Puede ser desde simplemente aspirar la casa como lo hacen los robots Rumba, o puede ser una tarea más compleja como la asistencia a un adulto mayor.

El Laboratorio de Robots Autónomos y Sistemas Cognitivos (ARCOSLab) se encuentra en la construcción de un robot humanoide completo, que trabajará como asistente a los humanos en tareas de la vida cotidiana. Este incluye las partes de: Cabeza, torso, hombros, brazos, manos, base omnidireccional, etc. Este tendrá la capacidad de poder cocinar un desayuno básico. 

Para incrementar las capacidades de asistencia del robot, es importante que el robot no este solo limitado a una mesa de trabajo, sino que también pueda tomar objetos del suelo y colocarlo en sitios elevados. Por lo que necesitará de una articulación que pueda desplazarse a la altura deseada.

Basado en un diseño preliminar este proyecto pretende desarrollar una articulación para el torso que le permita al robot aumentar el rango de alcance de objetos. También, utilizando resultados de investigaciones previas, se piensa determinar el montaje óptimo de los componentes para lograr un mejor aprovechamiento del hardware disponible. Finalmente, la articulación contara con control por impedancia que le permitirá un control amigable con los objetos y el ambiente a su alrededor. Con estas características el robot tendrá una articulación con una alta capacidad de manipulación en ambientes poco controlados y compartidos con humanos.


\section{Alcance del proyecto}
Se creará un torso y articulación para el robot humanoide del ARCOSLab con capacidades de control por impedancia en el eje z. Para que pueda detectar fuerzas externas en esa dirección. Para esto también es necesario la programación que mida e interprete las fuerzas externas.\\
Este podrá desplazarse en el eje z mediante el uso de un tornillo sin fin. Habilitándole alcanzar objetos que estén en alturas bajas y altas.\\


\section{Objetivos.}
Los objetivos de este proyecto son:

\subsection{Objetivo general}
Desarrollar mejoras al diseño y construir una articulación con control por impedancia para el robot humanoide del ARCOSLab. 

\subsection{Objetivos específicos}
Para el desarrollo de este proyecto se establecieron los siguientes objetivos:

\begin{itemize} % lista con viñetas
\item Analizar las fortalezas y debilidades del diseño actual mediante la recopilación de los diseños existentes.
\item Utilizando resultados de investigaciones previas determinar características deseables en la articulación del torso. 
\item Modificar al diseño actual que resuelva las debilidades actuales, asegurándose la implementación de las características deseables que incluyan control por impedancia para la articulación del torso del robot humanoide.
\item Determinar de la lista de componentes y especificaciones detalladas de cada uno de ellos.
\item Construir las partes de la articulación del torso.
\item Ensamblar la articulación y montarla en el torso.
\item Ejecutar pruebas de funcionamiento y analizar sus resultados.
\end{itemize}

\section{Metodología}
\begin{enumerate}  %lista numerada
	\item Revisión de boceto escaneado de las dimensiones de los componentes del torso (computadores, controladores, eje z del torso, altura final del torso).
	\item Análisis de fortalezas y debilidades del diseño anterior del torso.
	\item Diseño de un nuevo torso considerando investigaciones previas.
	\item Simulación del nuevo torso y modificación de este si es requerido.
	\item Investigación de control por impedancia.
	\item Creación de articulación que medirá fuerzas en el eje z para implementar control por impedancia.
	\item Programación del análisis para el control por impedancia.
	\item Ejecución de pruebas para el torso y la articulación con control por impedancia.
\end{enumerate}